Dron podwodny

Program sluzy obsludze i wizualizacji drona podwodnego.

Aby uruchomic program nalezy (jesli wczesniej nie bylo to robione) uruchomic program cmake komenda \+: \textquotesingle{} cmake . \textquotesingle{} Nastepnie gdy make zostal stworzony, kompilujemy program komenda \+: \textquotesingle{} make \textquotesingle{} . Na koniec uruchamiamy go poleceniem \textquotesingle{} ./dronv1 \textquotesingle{} (jesli nie jestesmy w folderze z dronem, nalezy kropke zastapic sciezka do programu). Nalezy zainstalowac program gnuplot, pomagajacy nam wizualizowanie ruchu drona

W programie mozemy poruszac sie dronem pomiedzy przeszkodami, kozystajac z menu uzytkownika. Dostepne opcje to\+:

-\/\+Obrot -\/\+Wznoszenie/opadanie pod zadanym katem -\/\+Wyjscie z programu

Ruch drona obrazuje animacja w programie gnuplot.

Uwagi\+:
\begin{DoxyEnumerate}
\item Nie wiem jak sprawic , zeby komendy skierowane do gnuplota nie wciskaly sie miedzy moj tekst a wczytywanie. Ne powoduje to bledow, ale wyglada zle \+:(
\item Zderzenie z woda i dnem nadal nie dziala, mimo iz dolozylem wszelkich staran, nie mam pojecia co jeszcze moze psuc wykrywanie kolizji 
\end{DoxyEnumerate}